\documentclass[10pt,a4paper]{article}
\usepackage[utf8x]{inputenc}
\usepackage{ucs}
\usepackage{amsmath}
\usepackage{amsfonts}
\usepackage{amssymb}
\usepackage[russian]{babel}
\usepackage{listings}
\author{Кевролетин В.В. 236гр.}
\title{$\lambda$-исчисление}
\begin{document}

\maketitle

\subsection*{Задание4}
\subsubsection*{Условие}
Показать, что ромбовидное сво-во не выполняется, если $\rightarrow $
заменить на $ \twoheadrightarrow $.
\subsubsection*{Решение}
Возьмем терм $(\lambda x.xx)((\lambda z.z)y)$ и убедимся, проделав все возможные
преобразования, что св-во не выполняется: \\
\begin{enumerate}
\item $(\lambda x.xx)((\lambda z.z)y) \longrightarrow_\beta (\lambda
  x.xx)y \longrightarrow_\beta yy$
\item $(\lambda x.xx)((\lambda z.z)y) \longrightarrow_\beta ((\lambda
  z.z)y)((\lambda z.z)y) \longrightarrow_\beta ((\lambda z.z)y)y
  \longrightarrow_\beta yy$
\item $(\lambda x.xx)((\lambda z.z)y) \longrightarrow_\beta ((\lambda
  z.z)y)((\lambda z.z)y) \longrightarrow_\beta y((\lambda z.z)y)
  \longrightarrow_\beta yy$
\end{enumerate}

\subsection*{Задание5}
\subsubsection*{Условие}
Доказать, что добавление аксиомы $\lambda xy.x = \lambda xy.y$,
получим $\forall P,Q: P=Q$
\subsubsection*{Решение}
Подставим в качестве аргументов $PQ$ в праую и левую абстракцию:\\
$(\lambda xy.x)PQ \longrightarrow_\beta P, $\\
$(\lambda xy.y)PQ \longrightarrow_\beta Q, $\\
но $(\lambda xy.x) = (\lambda xy.y)$, так что в силу свойств отношения
$=$ имеем: P=Q, а в силу произвольности выбора P и Q, получаем
равенство для $ \forall PQ $ \\

\subsection*{Задание6}
\subsubsection*{Условие}
Показать, что операция возведения в степень выглядят
следующим образом:\\
$expt \equiv \lambda mnfx.nmfx$
\subsubsection*{Решение}
$
(expt\ m\ n) \equiv (\lambda mnfx.nmfx)\ m \ n \rightarrow 
(\lambda fx.n(f^m)x) \rightarrow 
(\lambda fx.\underbrace{(f^m(f^m(... (f^m}_{n\ times} x)...)))  \equiv
(\lambda fx.f^{m^n}x)
\equiv m^n
$

\subsection*{Задание7}
\subsubsection*{Условие}
Показать, чт:о
\begin{enumerate}
\item $head(cons\ M\ N) \longrightarrow M$ 
\item $tail(cons\ M\ N) \longrightarrow N$ 
\end{enumerate}
\subsubsection*{Решение}
\begin{enumerate}
\item
 $head(cons\ M\ N) \longrightarrow head(pair\ false\ (pair\ M\
 N)) \longrightarrow (first\ (second\ (pair\ false\ (pair\ M\
 N)))) \longrightarrow (first\ (second\ (pair\ false\ (pair\ M\
 N)))) \longrightarrow (first\ (pair\ M\
 N)) \longrightarrow M $
\item
 $tail(cons\ M\ N) \longrightarrow tail(pair\ false\ (pair\ M\
 N)) \longrightarrow (second\ (second\ (pair\ false\ (pair\ M\
 N)))) \longrightarrow (second\ (second\ (pair\ false\ (pair\ M\
 N)))) \longrightarrow (second\ (pair\ M\
 N)) \longrightarrow M$ 
\end{enumerate}

% спасибо Васильевой Лене
\subsection*{Задание8}
\subsubsection*{Условие}
Представить натуральные числа при помощи списков. Ввести несколько операций.
\subsubsection*{Решение}
$0 = nil$\\
$1 = cons\ nil\ nil$\\
$n + 1 = cons\ n\ nil$\\
$n + m = \underbrace{cons (cons(}_{m\ times}...cons (n\ nil)... )
nil)$ \\
$n - 1 = tail\ n$\\
$n - m = \underbrace{tail (tail(}_{m\ times}...tail\  n)... )$ \\

\subsection*{Задание9}
\subsubsection*{Условие}
Функция Акермана имеет следующее рекурсивное определение: \\
$ack\ 0\ n = n + 1 \\
ack\ (m\,+\,1)\ 0 = ack\ m\ 1 \\
ack\ (m\,+\,1)\ (m\,+\,1) = ack\ m (ack\ (m\,+\,1)\ n)$\\
Показать, что функция Акермана в терминах $\lambda$ исчисления выглядит
следующим образом:
$ack \equiv \lambda m.m(\lambda fn.nf(f\ 1))suc$
\subsubsection*{Решение}
\begin{enumerate}
\item
$ack\ 0\ n \equiv (\lambda m.m(\lambda fn.nf(f\ 1))suc)\ 0\ n \rightarrow
0(\lambda fn.nf(f\ 1))suc\ n \rightarrow suc\ n \equiv (n\,+\,1)$
\item
$ack\ (m\,+\,1)\ n \rightarrow (\lambda m.m(\lambda fn.nf(f\ 1))suc)\
(m\,+\,1)\ n \rightarrow (m\,+\,1)(\lambda fn.nf(f\ 1))suc\ n
\rightarrow (\lambda fn.nf(f\ 1))(m\ \lambda fn.nf(f\ 1) suc) n = (\lambda
fn.nf(f\ 1))(ack\ m) n \rightarrow n(ack\ m)(ack\ m\ 1) $ 
\item
$ ack (m+1)\ 0 \rightarrow 0\ (ack\ m) (ack\ m\ 1) \rightarrow ack\
m\ 1  $
\item
$ack(m+1)(n+1) \rightarrow n+1\ (ack\ m)(ack\ m\ n) \rightarrow ack\
m(n(ack\ m)(ack\ m\ 1))) = ack\ m\  (ack(m+1)n)$
\end{enumerate}

\end{document}

\documentclass[10pt,a4paper]{article}
\usepackage[cm]{fullpage}
\usepackage[utf8x]{inputenc}
\usepackage{ucs}
\usepackage{amsmath}
\usepackage{amsfonts}
\usepackage{amssymb}
\usepackage[russian]{babel}
\usepackage{listings}
\author{Кевролетин В.В. 236гр.}
\title{$\lambda$-исчисление}
\begin{document}

\maketitle

\subsection*{Задание31}
Привести пример функции $f:N\rightarrow N$, которая обладает свойством
$f(1)+f(2)\not =f(2)+f(1)$. Объяснить причину такого поведения.
\subsubsection*{Условие}
\subsubsection*{Решение}
\begin{lstlisting}
(define count 0)

(define (f x)
  (set! count (* (+ count x) x))
  count)
\end{lstlisting}

Результат, возвращаемый описанной выше функции f(x) зависит от
глобальной переменной
count, значение которой меняется в теле этой же функции. Поэтому
результат функции зависит не только от переданного ей аргумента, но и
от
последовательности предыдущих вызовов f(x).
$f(1)+f(2)\not =f(2)+f(1)$ несмотря на коммутативность сложения,
потому что в первом случае сначала вычисляется f(1) а потом f(2), а во
втором случае наоборот.

\subsection*{Задание32}
\subsubsection*{Условие}
In the make-withdraw procedure, the local variable balance is created as a parameter of make-withdraw. We could also create the local state variable explicitly, using let, as follows:
\begin{lstlisting}
(define (make-withdraw initial-amount)
  (let ((balance initial-amount))
    (lambda (amount)
      (if (>= balance amount)
          (begin (set! balance (- balance amount))
                 balance)
          "Insufficient funds"))))
\end{lstlisting}
Show that the two versions of make-withdraw create objects with the same behavior. How do the environment structures differ for the two versions?
\subsubsection*{Решение}

\lstset{ %
basicstyle=\footnotesize,       % the size of the fonts that are used for the code
frame=single
}
\begin{lstlisting}

       +----------------------------------------------------+
global |                                                    |
env    | make-withdraw----+                                 |
       |                  |                                 |
       |                  |                                 |
       +------------------|---------------------------------+
                          |       ^
                          |      /|\
                          |       |
                         \|/      |
                          '       |
                 parameters: initial-amount
                 body:(let ((balance initial-amount))
                        (lambda (amount)
                          (if (>= balance amount)
                             (begin (set! balance (- balance amount))
                                    balance)
                             "Insufficient funds"))))

\end{lstlisting}

После выполнения (define W1 (make-withdraw 100)) будет создан объект
w1. Ниже на схеме изображено состояние окружения после сознания w1. На
схеме видно, что полученный объект не отличается от созданного в 1 версии.

\begin{lstlisting}

       +----------------------------------------------------+
global | make-withdraw ...                                  |   /
env    |                                                    |  <----------------+
       | w1---+                                             |   \               |
       |      |                                             |                   |
       +------|---------------------------------------------+                   |
              |                                                  +---------+    |
              |    +----------------------------------------+    |         |    |
              |    |                                       \|/   |        \|/   |
             \|/   |                                        '    |         '    |
              '    |                                      +---------+ +---------------+
 parameters: initial-amount                               | balance:| | initial-amount|
 body:                                                    |   100   | |      100      |
                                                          +---------+ +---------------+
       (if (>= balance amount)
          (begin (set! balance (- balance amount))
                 balance)
          "Insufficient funds"))))

\end{lstlisting}

\lstset{ %
basicstyle=\footnotesize,       % the size of the fonts that are used for the code
frame=none
}


\subsection*{Задание33}
\subsubsection*{Условие}
make-cycle
\subsubsection*{Решение}

Было:
\begin{lstlisting}

      +---+---+       +---+---+       +---+---+
z-->  |   |   +---->  |   |   +---->  |   |   +---->  nil
      +-+-+---+       +-+-+---+       +-+-+---+
        |               |               |
       \|/             \|/             \|/
        '               '               '
        a               b               c

\end{lstlisting}

Станет:
\begin{lstlisting}

        +------------------------------------------+
        |                                          |
       \|/                                         |
        '                                          |
      +---+---+       +---+---+       +---+---+    |
z-->  |   |   +---->  |   |   +---->  |   |   +----+
      +-+-+---+       +-+-+---+       +-+-+---+
        |               |               |
       \|/             \|/             \|/
        '               '               '
        a               b               c

\end{lstlisting}

Вызов (last-pair z) приведёт к зацикливанию выполнения программы,
т.к. условие (null? x) не выполняется не для одного элемента списка.

\subsection*{Задание34}
\subsubsection*{Условие}
mystery
\begin{lstlisting}
(define (mystery x)
  (define (loop x y)
    (if (null? x)
        y
        (let ((temp (cdr x)))
          (set-cdr! x y)
          (loop temp x))))
  (loop x '()))
\end{lstlisting}
\subsubsection*{Решение}
Процедура изменяет переданный ей агрумент-список и позвращает его в
качестве результата. Изменения проще всего показать на примере: пусть
сначала мы имеем список (a b c d) тогда последовательно к этой
последовательности будут добавлены все её несобственные  суффиксы
начиная с самого длинного: \\
(a b c d)               ;; исходная последовательность \\
(a b c d b c d)         ;; добавили (b c d) \\
(a b c d b c d c d)     ;; добавили (с d) \\
(a b c d b c d c d d)   ;; добавили (d) \\

\begin{lstlisting}

       +---+---+       +---+---+       +---+---+       +---+---+
v-->   |   |   +---->  |   |   +---->  |   |   +---->  |   |   +---->  nil
       +-+-+---+       +-+-+---+       +-+-+---+       +-+-+---+
         |               |               |               |
        \|/             \|/             \|/             \|/
         '               '               '               '
         a               b               c               d


       +---+---+       +---+---+       +---+---+       +---+---+
v-->   |   |   +---->  |   |   +---->  |   |   +---->  |   |   +---+
       +-+-+---+       +-+-+---+       +-+-+---+       +-+-+---+   |
         |               |               |               |         |
        \|/             \|/             \|/             \|/        |
         '               '               '               '         |
         a               b               c               d         |
                +--------------------------------------------------+
                |
                |      +---+---+       +---+---+       +---+---+
                +--->  |   |   +---->  |   |   +---->  |   |   +---+
                       +-+-+---+       +-+-+---+       +-+-+---+   |
                         |               |               |         |
                        \|/             \|/             \|/        |
                         '               '               '         |
                         b               c               d         |
                               +-----------------------------------+
                               |
                               |       +---+---+       +---+---+
                               +---->  |   |   +---->  |   |   +---+
                                       +-+-+---+       +-+-+---+   |
                                         |               |         |
                                        \|/             \|/        |
                                         '               '         |
                                         c               d         |
                                               +-------------------+
                                               |
                                               |       +---+---+
                                               +---->  |   |   +---->  nil
                                                       +-+-+---+
                                                         |
                                                        \|/
                                                         '
                                                         d


\end{lstlisting}

\subsection*{Задание35}
\subsubsection*{Условие}
count-pairs             
\subsubsection*{Решение}
3                       
\begin{lstlisting}      
                        
 +---+---+       +---+---+       +---+---+
 |   |   +---->  |   |   +---->  |   |   +---->  nil
 +-+-+---+       +-+-+---+       +-+-+---+
   |               |               |
  \|/             \|/             \|/
   '               '               '
   a               b               c
                        
\end{lstlisting}        
4                       
\begin{lstlisting}      
                   +---------------+
                   |              \|/
                   |               '
 +---+---+       +-+-+---+       +---+---+
 |   |   +---->  |   |   +---->  |   |   +---->  nil
 +-+-+---+       +-+-+---+       +-+-+---+
   |                               |
  \|/                             \|/
   '                               '
   a                               c
                        
\end{lstlisting}        
7                       
\begin{lstlisting}      
                        
                   +---------------+
                   |              \|/
                   |               '
 +---+---+       +-+-+---+       +---+---+
 |   |   +---->  |   |   +---->  |   |   +---->  nil
 +-+-+---+       +---+---+       +-+-+---+
   |               ^               |
   |              /|\             \|/
   +---------------+               '
                                   c
                        
\end{lstlisting}        
Никогда не завершится   
\begin{lstlisting}      
   +----------------------------------------+
  \|/                                       |
   '                                        |
 +---+---+       +---+---+       +---+---+  |
 |   |   +---->  |   |   +---->  |   |   +--+
 +-+-+---+       +-+-+---+       +-+-+---+
   |               |               |
  \|/             \|/             \|/
   '               '               '
   a               b               c
                        
\end{lstlisting}        
                        
                        
                        
\subsection*{Задание}   
\subsubsection*{Условие}
\subsubsection*{Решение}
\begin{lstlisting}
\end{lstlisting}

\subsection*{Задание}   
\subsubsection*{Условие}
\subsubsection*{Решение}
\begin{lstlisting}
\end{lstlisting}

\subsection*{Задание}   
\subsubsection*{Условие}
\subsubsection*{Решение}
\begin{lstlisting}
\end{lstlisting}


\end{document}

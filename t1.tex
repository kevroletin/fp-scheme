\documentclass[10pt,a4paper]{article}
\usepackage[utf8x]{inputenc}
\usepackage{ucs}
\usepackage{amsmath}
\usepackage{amsfonts}
\usepackage{amssymb}
\usepackage[russian]{babel}
\author{Кевролетин В.В. 236гр.}
\title{$\lambda$-исчисление}
\begin{document}

\maketitle

\subsection*{Задание1}
\subsubsection*{Условие}
Подробно расписать преобразования:\\
$(\lambda xy.L)MN$, \textit{где $y$ свободно входит в $M$}
\subsubsection*{Решение}

\begin{enumerate}
  \item $(\lambda xy.L) \longrightarrow_\alpha (\lambda xz.L[z|y])$
  \item $(\lambda xz.L[z|y]) = (\lambda x.(\lambda z.L[z|y])M)N \longrightarrow_\beta
(\lambda x.L[z|y][M|z])N$
  \item $(\lambda x.L[z|y][M|z])N \longrightarrow_\beta L[z|y][M|z][N|x]$
\end{enumerate}

\subsection*{Задание2}
\subsubsection*{Условие}
Привести к нормальной форме 2мя способами:\\
$(\lambda x.(\lambda y.xy)z)y$
\subsubsection*{Решение}

\begin{enumerate}
  \item $(\lambda x.(\lambda y.xy)z)y \longrightarrow_\eta (\lambda
    x.xz)y \longrightarrow_\beta yz$
  \item $(\lambda x.(\lambda y.xy)z)y \longrightarrow_\beta (\lambda
    x.xz)y \longrightarrow_\beta yz$
\end{enumerate}

\subsection*{Задание3}
\subsubsection*{Условие}
Привести к нормальной форме:\\
$(\lambda x.xxy)(\lambda x.xxy)$
\subsubsection*{Решение}
$(\lambda x.xxy)(\lambda x.xxy) \longrightarrow_\beta  (\lambda
x.xxy)(\lambda x.xxy)y \longrightarrow_\beta (\lambda
x.xxy)yyy \longrightarrow_\beta yyyy$

\end{document}

\documentclass[10pt,a4paper]{article}
\usepackage[utf8x]{inputenc}
\usepackage{ucs}
\usepackage{amsmath}
\usepackage{amsfonts}
\usepackage{amssymb}
\usepackage[russian]{babel}
\author{Кевролетин В.В. 236гр.}
\title{$\lambda$-исчисление (2 задание)}
\begin{document}

\maketitle

\subsection*{Задание1}
\subsubsection*{Условие}
Показать, что ромбовидное сво-во не выполняется, если $\rightarrow $
заменить на $ \twoheadrightarrow $.
\subsubsection*{Решение}
Возьмем терм $(\lambda x.xx)((\lambda z.z)y)$ и убедимся, проделав все возможные
преобразования, что св-во не выполняется: \\
\begin{enumerate}
\item $(\lambda x.xx)((\lambda z.z)y) \longrightarrow_\beta (\lambda
  x.xx)(y) \longrightarrow_\beta (y)(y)$
\item $(\lambda x.xx)((\lambda z.z)y) \longrightarrow_\beta ((\lambda
  z.z)y)((\lambda z.z)y) \longrightarrow_\beta ((\lambda z.z)y)(y)
  \longrightarrow_\beta (y)(y)$
\item $(\lambda x.xx)((\lambda z.z)y) \longrightarrow_\beta ((\lambda
  z.z)y)((\lambda z.z)y) \longrightarrow_\beta (y)((\lambda z.z)y)
  \longrightarrow_\beta (y)(y)$
\end{enumerate}

\subsection*{Задание2}
\subsubsection*{Условие}
Доказать, что добавление аксиомы $\lambda xy.x = \lambda xy.y$,
получим $\forall P,Q: P=Q$
\subsubsection*{Решение}
Подставим в качестве аргументов $PQ$ в праую и левую абстракцию:\\
$(\lambda xy.x)PQ \longrightarrow_\beta P, $\\
$(\lambda xy.y)PQ \longrightarrow_\beta Q, $\\
но $(\lambda xy.x) = (\lambda xy.y)$, так что в силу свойств отношения
$=$ имеем: P=Q, а в силу произвольности выбора P и Q, получаем
равенство для $ \forall PQ $ \\

\subsection*{Задание3}
\subsubsection*{Условие}
Показать, что операции умножения и возведения в степень выглядят
следующим образом:\\
$mult \equiv \lambda mnfx.m(nf)x$\\
$expt \equiv \lambda mnfx.mnfx$
\subsubsection*{Решение}
\begin{enumerate}
\item 
$(mult\ m\ n) \longrightarrow \lambda fx.m(nf)x \longrightarrow 
\lambda fx.\underbrace{nf(nf(\ldots \overbrace
{nf(nfx}^\text{=n+n}}_\text{n times})\ldots) = \lambda
fx.\underbrace{f(f(\ldots f(fx}_{n \cdot m\ times})\ldots) \equiv
n\cdot m$ 

\item
ММИ:
База:\\
$(expt\ 0\ n) \longrightarrow \lambda fx.n0fx \longrightarrow \lambda
fx.0f(0f(\ldots 0f)\ldots )x \longrightarrow \lambda fx.x \equiv 0 $ \\
$(expt\ m\ n) \longrightarrow \lambda fx.0mfx \longrightarrow =>
\lambda fx.fx \equiv 1$ \\
Переход:\\
$(expt\ m\ {n\!+\!1}) = (mult\ (expt\ m\ n)\ m) \longrightarrow (mult\
(\lambda fx.nmfx)\ m) \longrightarrow \lambda fx.(\lambda
fx.nmfx)(mf)x \longrightarrow \lambda fx.nm\bold{(mf)}x \longrightarrow
\lambda fx.\overbrace{\underbrace{m(m(m(\ldots m}_{n\
    times}\bold{(mf)}}^{n+1\ times}\ldots )x \longrightarrow^{-1}
\lambda fx.{n\!+\!1}\, mfx \equiv  (expt\ m\ n+1)$
\end{enumerate}

\subsection*{Задание4}
\subsubsection*{Условие}
Показать: $(\lambda mn.m\,suc\,n)mn \longrightarrow m+n $
\subsubsection*{Решение}
 $(\lambda mn.m\,suc\,n)mn \longrightarrow m\,suc\,n
\longrightarrow \underbrace{ suc(suc(suc \ldots suc(}_{m\ times}n)
\ldots ) \longrightarrow \underbrace{ suc(suc \ldots suc(}_{m-1\
  times}{n\!+\!1}) \ldots ) \longrightarrow n + m $
\subsection*{Задание5}
\subsubsection*{Условие}
Показать, чт:о
\begin{enumerate}
\item $head(cons\ M\ N) \longrightarrow M$ 
\item $tail(cons\ M\ N) \longrightarrow N$ 
\end{enumerate}
\subsubsection*{Решение}
\begin{enumerate}
\item
 $head(cons\ M\ N) \longrightarrow head(pair\ false\ (pair\ M\
 N)) \longrightarrow (first\ (second\ (pair\ false\ (pair\ M\
 N)))) \longrightarrow (first\ (second\ (pair\ false\ (pair\ M\
 N)))) \longrightarrow (first\ (pair\ M\
 N)) \longrightarrow M $
\item
 $tail(cons\ M\ N) \longrightarrow tail(pair\ false\ (pair\ M\
 N)) \longrightarrow (second\ (second\ (pair\ false\ (pair\ M\
 N)))) \longrightarrow (second\ (second\ (pair\ false\ (pair\ M\
 N)))) \longrightarrow (second\ (pair\ M\
 N)) \longrightarrow M$ 
\end{enumerate}

\subsection*{Задание6}
\subsubsection*{Условие}
Представить натуральные числа при помощи списков. Ввести несколько операций.
\subsubsection*{Решение}
$0 - false \equiv \lambda z.z \\
1 - (pair\ true\ false) \\
\ldots \\
n - (pair\ true\ {n\!-\!1})\\
\\
iszero \equiv \lambda z.(not\ (first\ z)) \\
suc \equiv \lambda z.(pair\ true\ z) $ \\
арифметеческие операции можно реализовать рекурсивными функциями



\subsection*{Задание7}
\subsubsection*{Условие}
Функция Акермана имеет следующее рекурсивное определение: \\
$ack\ 0\ n = n + 1 \\
ack\ (m\,+\,1)\ 0 = ack\ m\ 1 \\
ack\ (m\,+\,1)\ (m\,+\,1) = ack\ m (ack\ (m\,+\,1)\ n)$\\
Показать, что функция Акермана в терминах $\lambda$ исчисления выглядит
следующим образом:
$ack \equiv \lambda m.m(\lambda fn.nf(f1))suc$
\subsubsection*{Решение}
\begin{enumerate}
\item
$ack\ 0\ n \equiv \lambda m.m(\lambda fn.nf(f1))suc\ 0\ n \rightarrow
0(\lambda fn.nf(f1))suc\ n \rightarrow suc n \equiv (n\,+\,1)$
\item
$ack\ (m\,+\,1)\ n \rightarrow (\lambda m.m(\lambda fn.nf(f1))suc)\
(m\,+\,1)\ n \rightarrow (m\,+\,1)(\lambda fn.nf(f1))suc\ n
\rightarrow (\lambda fn.nf(f1))(m\ \lambda fn.nf(f1) suc) n = (\lambda
fn.nf(f1))(ack m) n \rightarrow n(ack\ m)(ack\ m\ 1) $ \item
$ ack (m+1)\ 0 \rightarrow 0 (ack\ (m+1) (ack\ m\ 1) \rightarrow ack\
m\ 1  $
\item
$ack(m+1)(n+1) \rightarrow n+1\ (ack\ m)(ack\ m\ n \rightarrow ack\
m(n(ack\ m)(ack\ m\ 1))) = ack (ack(m+1)n)$
\end{enumerate}

\end{document}

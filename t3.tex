\documentclass[10pt,a4paper]{article}
\usepackage[utf8x]{inputenc}
\usepackage{ucs}
\usepackage{amsmath}
\usepackage{amsfonts}
\usepackage{amssymb}
\usepackage[russian]{babel}
\author{Кевролетин В.В. 236гр.}
\title{$\lambda$-исчисление (3 задание)}
\begin{document}

\maketitle

\subsection*{Задание1}
\subsubsection*{Условие}
Показать, что $\Theta{}$ - оператор неподвижной точки 
\subsubsection*{Решение}

$A \equiv \lambda xy.y(xxy) \\
\Theta \equiv AA \\
\Theta \ F = \lambda xy.y(xxy)\lambda xy.y(xxy)F \rightarrow F(\lambda
xy.y(xxy)\lambda xy.y(xxy)F) = F \Theta F $

\subsection*{Задание2}
\subsubsection*{Условие}
Показать, что $(first (second (zeros \ 1))) \twoheadrightarrow 0 $
\subsubsection*{Решение}
$
zeros1 \equiv \Theta (pair\ 0) \twoheadrightarrow (pair\ 0 (pair\ 0 (
\ldots \Theta (pair\ 0) \ldots ))) \equiv (pair\ 0\ zeros1)\\
(first (second (zeros1))) = (first (second (pair (1 \ zeros))))
\rightarrow  (first (second(\lambda f.f(1 \ zeros)))) \rightarrow
(first ((\lambda p.p \ false)\lambda f.f(1 \ zeros))) \rightarrow
(first ((\lambda f.f(1 \ zeros ) false))) \rightarrow (first (false\
1\ zeros)) \rightarrow (first\ zeros) \rightarrow 1$

\subsection*{Задание3}
\subsubsection*{Условие}
Определены ли термы? \\
Y\\
Y not\\
K\\
YI\\
x$\Omega$\\
YK\\
Y(Kx)\\
n\\
\subsubsection*{Решение}
\begin{enumerate}
\item
$Y \equiv \lambda f.(\lambda x.f(xx))(\lambda x.f(xx)) \rightarrow
\lambda f.f(\lambda x.f(xx))(\lambda x.f(xx)) \rightarrow \ldots $\\
не определён
\item
$Y\ not \rightarrow not\ Y\ not \rightarrow \ldots$ \\
не определён
\item
$K \equiv \lambda xy.x$ - в головной нормальной форме\\
определён
\item
$Y\,I \rightarrow I\ Y\ I \rightarrow Y\ I\ \rightarrow \ldots$\\
не определён
\item
$x \Omega $ - $\Omega$ не имеет нормальной формы\\
не определён
\item
$YK \rightarrow K(YK) \rightarrow K(K(YK)) \rightarrow \ldots  $\\
не определён
\item
$Y(Kx) \rightarrow Kx(Y(Kx)) \rightarrow x$\\
определён
\item
$n$ \\
определён
\end{enumerate}

\end{document}

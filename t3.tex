\documentclass[10pt,a4paper]{article}
\usepackage[utf8x]{inputenc}
\usepackage{ucs}
\usepackage{amsmath}
\usepackage{amsfonts}
\usepackage{amssymb}
\usepackage[russian]{babel}
\author{Кевролетин В.В. 236гр.}
\title{$\lambda$-исчисление}
\begin{document}

\maketitle

\subsection*{Задание10}
\subsubsection*{Условие}
Показать, что $\Theta{}$ - оператор неподвижной точки 
\subsubsection*{Решение}

$A \equiv \lambda xy.y(xxy) \\
\Theta \equiv AA \\
\Theta \ F = \lambda xy.y(xxy)\lambda xy.y(xxy)F \rightarrow F(\lambda
xy.y(xxy)\lambda xy.y(xxy)F) = F \Theta F $

\subsection*{Задание11}
\subsubsection*{Условие}
Выяснить, разрешимы и определены ли термы? \\
Y\\
Y not\\
K\\
YI\\
x$\Omega$\\
YK\\
Y(Kx)\\
n\\
\subsubsection*{Решение}

Терм определён тогда и только тогда, когда он может быть редуцирован к
головной нормальной форме; иначе неопределён. \\
Терм M разрешим тогда и только тогда, когда существуют переменные $x_1,
... x_m $ и термы $N_1, ... N_n$ такие, что $(\lambda x_1
... x_m.M)N_1 ... N_n = I $ \\
Сперва отметим, что: \\
 - если M = xM, то M не имеет нормальной формы. \\
 - если M = PM, то M не имеет нормальной формы, если $ P \not = I $ и \\
 P не функция 1 аргумента, возвращающая M. \\
 - все, что определено при помощи Y не имеет нормальной формы.

\begin{enumerate}
\item
$Y \equiv \lambda f.(\lambda x.f(xx))(\lambda x.f(xx)) \rightarrow
\lambda f.f(\lambda x.f(xx))(\lambda x.f(xx)) \rightarrow \ldots $\\
не определён \\
$Y\lambda z.y \rightarrow (\lambda z.y)(Y\lambda z.y) \rightarrow y $ \\
$(\lambda y.y)I \rightarrow I$
разрешим 
\item
$Y\ not \rightarrow not\ ( Y\ not) \rightarrow \ldots$ \\
не определён \\
$Y\ not\ N \rightarrow not(Y\ not)\ N \rightarrow \ldots $ \\
не разрешим
\item
$K \equiv \lambda xy.x$ - в головной нормальной форме\\
определён \\
$K\ I\ N= (\lambda xy.x)\ I\ N \rightarrow I$
разрешим
\item
$Y\,I \rightarrow I\ (Y\ I) \rightarrow Y\ I\ \rightarrow \ldots$\\
не определён \\
$Y\,I\ N \rightarrow I\ (Y\ I) N \rightarrow Y\ I\ N \rightarrow \ldots$\\
не разрешим
\item
$x \Omega $ \\
$\Omega$ не имеет нормальной формы\\
не определён \\
$(\lambda x.x \Omega)\lambda z.I \rightarrow I $ \\
разрешим
\item
$YK \rightarrow K(YK) \rightarrow K(K(YK)) \rightarrow \ldots  $\\
не определён \\
$YK x_1x_2...x_n \rightarrow 
K(YK) x_1x_2...x_n \rightarrow 
K(K(YK)) x_1x_2...x_n  \rightarrow  
K(K(YK)) x_2...x_n  \rightarrow
\ldots \rightarrow
YK
$\\
не разрешим
\item
$Y(Kx)$ определён с использованием оператора Y\\
не определён \\
$Y(Kx) \rightarrow Kx(Y(Kx)) \rightarrow x$\\
$(\lambda x.x)I \rightarrow I$
разрешим
\item
$n$ - в нормальной форме \\
определён \\
$nII \equiv (\lambda fx.f^nx)II \rightarrow I^nI \rightarrow I $ \\
разрешим

\end{enumerate}

\subsection*{Задание12}
\subsubsection*{Условие}
Показать, что $(first (second (zeros \ 1))) \twoheadrightarrow 0 $
\subsubsection*{Решение}
$
zeros1 \equiv \Theta (pair\ 0) \twoheadrightarrow (pair\ 0 (pair\ 0 (
\ldots \Theta (pair\ 0) \ldots ))) \equiv (pair\ 0\ zeros1)\\
(first (second (zeros1))) = (first (second (pair (1 \ zeros))))
\rightarrow  (first (second(\lambda f.f(1 \ zeros)))) \rightarrow
(first ((\lambda p.p \ false)\lambda f.f(1 \ zeros))) \rightarrow
(first ((\lambda f.f(1 \ zeros ) false))) \rightarrow (first (false\
1\ zeros)) \rightarrow (first\ zeros) \rightarrow 1$


\end{document}

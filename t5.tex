\documentclass[10pt,a4paper]{article}
\usepackage[utf8x]{inputenc}
\usepackage{ucs}
\usepackage{amsmath}
\usepackage{amsfonts}
\usepackage{amssymb}
\usepackage[russian]{babel}
\usepackage{listings}
\author{Кевролетин В.В. 236гр.}
\title{Scheme}
\begin{document}

\maketitle

\subsection*{Задание18}
\subsubsection*{Условие}
map, append, length через accumulate
\subsubsection*{Решение}
\begin{lstlisting}
(define (map p sequence)
  (accumulate (lambda (x y) (cons (p x) y)) '() sequence))

(define (append seq1 seq2)
  (accumulate cons seq2 seq1))

(define (length sequence)
  (accumulate (lambda (x y) (+ 1 y)) 0 sequence))
\end{lstlisting}

\subsection*{Задание19}
\subsubsection*{Условие}
accumulate-n 
\subsubsection*{Решение}
\begin{lstlisting}
(define (accumulate-n op init seqs)
  (if (null? (car seqs))
      '()
      (cons (accumulate op init (map car seqs))
            (accumulate-n op init (map cdr seqs)))))
\end{lstlisting}

\subsection*{Задание20}
\subsubsection*{Условие}
dot-product, matrix-*-vector, transpose mat, matrix-*-matrix 
\subsubsection*{Решение}
\begin{lstlisting}
(define (dot-product v w)
  (accumulate + 0 (map * v w)))

(define (matrix-*-vector m v)
  (map (lambda (x) (dot-product x v)) m))

(define (transpose mat)
  (accumulate-n (lambda (x y) (cons x y)) '() mat))

(define (matrix-*-matrix m n)
  (let ((cols (transpose n)))
    (map (lambda (row) (matrix-*-vector n row) ) m)))
\end{lstlisting}

\subsection*{Задание21}
\subsubsection*{Условие}
fold-right and fold-left
\subsubsection*{Решение}
\begin{lstlisting}
(define (fold-right op initial sequence)
  (if (null? sequence)
      initial
      (op (car sequence)
          (fold-right op initial (cdr sequence)))))

(define (fold-left op initial sequence)
  (define (iter result rest)
    (if (null? rest)
        result
        (iter (op result (car rest))
              (cdr rest))))
  (iter initial sequence))
\end{lstlisting}

\subsection*{Задание22}
\subsubsection*{Условие}
reverse через fold-right and fold-left 
\subsubsection*{Решение}
\begin{lstlisting}
(define (reverse sequence)
  (fold-right (lambda (x y) (append y (list x))) '() sequence))

(define (reverse sequence)
  (fold-left (lambda (x y) (cons y x)) '() sequence))
\end{lstlisting}

\subsection*{Задание23}
\subsubsection*{Условие}
Two lists are said to be equal? if they contain equal elements arranged in the same order.
\subsubsection*{Решение}
\begin{lstlisting}
(define (equal? a b)
  (cond
   ((and (pair? a) (pair? b))
    (and (eq? (car a) (car b)) (equal? (cdr a) (cdr b))))
   ((and (not (pair? a)) (not (pair? b)))
    (eq? a b))
   (else '())))
\end{lstlisting}

\subsection*{Задание24}
\subsubsection*{Условие}
implement the differentiation rule for $u^n$... 
\subsubsection*{Решение}
\begin{lstlisting}

(define (deriv exp var)
  (cond ...
        ((power? exp)
         (make-product
           (make-power (power-get-arg exp) (- (power-get-pow exp) 1))
           (deriv (power-get-arg exp) var)))
        ...)

(define (make-power num pow) (list '^ num pow))

(define (power? x)
  (and (pair? x) (eq? (car x) '^)))

(define (power-get-arg p) (cadr p))
(define  (power-get-pow p)  (caddr p))

;;usage
(deriv (make-power (make-sum 'x 1) 2) 'x) 
;; > (* (^ (+ x 1) 1) (+ 1 0))

\end{lstlisting}

\subsection*{Задание25}
\subsubsection*{Условие}
Extend the differentiation program to handle sums and products of arbitrary numbers of (two or more) terms
\subsubsection*{Решение}
\begin{lstlisting}

(define (deriv exp var)
  (cond ...
        ((sum? exp)
         (cons '+
               (foldr (lambda (x y) (cons (deriv x var) y))
                      '()  (sum-args exp))))
        ((product? exp)
         (make-sum
           (make-product (deriv (product-first-arg exp) var)
                         (product-last-args exp))
           (make-product (product-first-arg exp)
                         (deriv (product-last-args exp) var))))
        ...)

(define (make-sum a1 . a2) (append (list '+ a1) a2))
(define (sum-args s) (cdr s))

(define (make-product m1 . m2) (append (list '* m1) m2))
(define (product-args p) (cdr p))
(define (product-first-arg s) (car (product-args s)))
(define (product-last-args s)
  (let ((tail (product-args s)))
    (if (> (length tail) 2)
        (cons '* tail))
        (car tail))))

;;usage

(define s (make-sum 1 2 3))
(sum-args s)                              ;; > (1 2 3)

(define p (make-product 1 2 3))
(product-args p)                          ;; > (1 2 3)
(product-first-arg p)                     ;; > 1
(product-last-args p)                     ;; > (* 2 3)
(product-last-args (product-last-args p)) ;; > 3

(deriv (make-sum 'x 'x 1) 'x)             ;; > (+ 1 1 0)
(deriv (make-product 'x 'x) 'x)           ;; > (+ (* 1 x) (* x 1))

\end{lstlisting}

\subsection*{Задание26}
\subsubsection*{Условие}
Suppose we want to modify the differentiation program so that it works with ordinary mathematical notation, in which + and * are infix rather than prefix operators
a. Show how to do this in order to differentiate algebraic expressions
presented in infix form, such as (x + (3 * (x + (y + 2)))). To
simplify the task, assume that + and * always take two arguments and
that expressions are fully parenthesized.
b. The problem becomes substantially harder if we allow standard algebraic notation, such as (x + 3 * (x + y + 2)), which drops unnecessary parentheses and assumes that multiplication is done before addition. Can you design appropriate predicates, selectors, and constructors for this notation such that our derivative program still works? 
\subsubsection*{Решение}
а)
\begin{lstlisting}
(define (make-sum a1 a2) (list a1 '+ a2))
(define (make-product m1 m2) (list m1 '* m2))

(define (sum? x)
  (and (pair? x) (eq? (cadr x) '+)))
(define (addend s) (car s))
(define (augend s) (caddr s))

(define (product? x)
  (and (pair? x) (eq? (cadr x) '*)))
(define (multiplier p) (car p))
(define (multiplicand p) (caddr p))

\end{lstlisting}
б) Ответ: нет, нельзя придумать предикаты и селекторы, так чтобы наша
программа работала. Можно перевести входные данные в префексную
нотацию и запустить на преобразованных данных наш алгоритм, или
использовать алгоритм рекурсивного спуска для разбора подобных выражений.

\end{document}
